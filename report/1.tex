\chapter{ПРАКТИЧЕСКАЯ РАБОТА № 1}

\section{Тема}

Генерация кода для выражения с константами и переменными.

\section{Задание}

Реализовать генерацию ассемблерного кода для арифметического выражения из констант и переменных.

\section{Ход выполнения работы}

В ходе выполнения работы был реализован модуль транслятора, осуществляющий генерацию $asm$-кода для арифметических выражений, состоящих из целочисленных констант и операций сложения, вычитания, умножения, целочисленного деления, взятия остатка от деления. Разработка опирается на лексический и синтаксический анализатор, реализованные в результате освоения дисциплины <<Теория автоматов и формальных языков>>. Входными данными для модуля является абстрактное синтаксическое дерево ($AST$), получаемое в результате лексического и синтаксического разбора математических выражений, записанных в текстовом виде. Выходными данными является $asm$-код, представляющий собой код вычисления выражения, поданного в качестве входных данных.

Генерация кода для узлов $AST$ выполняется по правилам, описанным ниже. Для узлов, представляющих константу --- команда ассемблера $PUSH$ $QWORD <n>$, где $<n>$ --- константа, хранящаяся в узле. Для узлов с бинарной операцией --- команды ассемблера:

\begin{itemize}
	\item{$POP$ $RBX$; второй операнд бинарной операции --- сверху стека;}
	\item{$POP$ $RAX$; первый операнд бинарной операции --- под вторым;}
	\item{Код конкретной операции;}
	\item{$PUSH$ $RAX$; помещение результата операции на вершину стека.}

Ниже приведен фрагмент кода обхода $AST$.

\begin{Verbatim}
	\linespread{1.0}
	public static void generateASM(Set<String> vars) {
		for (String var: vars) {
			System.out.println("MOV RCX, promt_" +  var);
			System.out.println("MOV R11, printf");
			System.out.println("CALL R11");

			System.out.println(" ");
			System.out.println("MOV RDX, scanf_format");
			System.out.println("MOV RDX, " + var);
			System.out.println("MOV R11, scanf");
			System.out.println("CALL R11\n");
		}
	}

	public static void main(String[] args) {
		String text = "x + y + 2";
		
		Lexer l = new Lexer(text);
		List<Token> tokens = l.lex();
		tokens.removeIf(t -> t.type == TokenType.SPACE);

		Parser p = new Parser(tokens);
		ExprNode node = p.parseExpression();
		Set<String> vars = new LinkedHashSet<>();
		getVars(vars, node);

		System.out.println("section .text\n" + 
			" global main\n" +
			" extern printf\n" + 
			" extern scanf\n" +
			" \n" +
			"main: ");
	}
\end{Verbatim}
